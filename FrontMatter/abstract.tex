Machine learning-based network intrusion detection systems (NIDS) can detect novel and evolving cyber threats by learning patterns from network traffic data. This dissertation applies machine learning classifiers, such as random forests and support vector machines (SVMs), to flow-based features extracted from network traffic. It investigates the transferability and generalisability of learned patterns across different datasets, simulating real-world deployment scenarios.

\noindent The key contributions are twofold: (1) analysing statistical properties of flow-based features to uncover characteristics distinguishing malicious and benign traffic, emphasising dataset representativeness and biases; (2) employing explainable AI techniques, such as SHAP, to identify the most relevant features for detecting intrusions and building trust in NIDS.\@

\noindent The findings contribute to developing effective NIDS solutions by shedding light on the transferability of learned patterns, dataset characteristics, and explainability. This dissertation guides the selection of representative training data and informs the design of robust and adaptable machine learning algorithms for network intrusion detection.