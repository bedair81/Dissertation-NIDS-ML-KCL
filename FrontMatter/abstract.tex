\setlength{\parindent}{0pt}

This dissertation investigates the effectiveness of machine learning-based Network Intrusion Detection Systems (NIDS) in detecting cyber threats by analysing patterns in network traffic data. The study assesses the transferability of Random Forest and Support Vector Machine (SVM) classifiers across the CTU13 and CICIDS2017 datasets. Models trained on one dataset are tested on the other to evaluate how well learned patterns generalise to different network environments. Additionally, the SHAP (SHapley Additive exPlanations) technique is employed to identify critical features influencing the detection of malicious traffic, enhancing the transparency of the models’ decision-making processes.

The novelty of this work lies in its comprehensive analysis of model transferability within NIDS and the application of explainable AI to improve interpretability. Findings reveal significant challenges arising from dataset biases and the limitations of directly applying models across datasets. To address these, the study proposes strategies to enhance detection accuracy and model interpretability, strengthening NIDS robustness against evolving cyber threats. This research advances cybersecurity by providing insights into the adaptability of machine learning models and the key factors impacting NIDS performance across diverse network settings.