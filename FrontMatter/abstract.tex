Machine learning-based network intrusion detection systems (NIDS) have emerged as a promising solution to detect novel and evolving cyber threats. By learning patterns and anomalies from network traffic data, algorithms such as random forests and support vector machines (SVMs) can identify previously unseen intrusions. This research focuses on the application of machine learning classifiers in NIDS, utilizing flow-based features extracted from network traffic.\@ The study investigates the transferability and generalizability of learned patterns across different datasets, simulating the real-world scenario of training a model on one dataset and deploying it to detect intrusions in the wild. By analyzing the statistical properties and distributions of flow-based features, this research aims to uncover the key characteristics that distinguish malicious traffic from benign traffic, emphasizing the importance of dataset representativeness and biases.\@ Furthermore, this research employs explainable AI techniques, such as SHAP (SHapley Additive exPlanations), to identify the most relevant features contributing to the detection of network intrusions. By understanding which features are crucial for accurate predictions, this study can gain insights into the underlying patterns and behaviors that indicate malicious activities. This explainability aspect is essential for reasoning about the model's decisions and building trust in the NIDS.\@ The findings of this study contribute to the development of effective and practical NIDS solutions by shedding light on the transferability of learned patterns, the impact of dataset characteristics, and the importance of explainability. By addressing these aspects, this research aims to guide the selection and generation of representative training data and inform the design of more robust and adaptable machine learning algorithms for network intrusion detection.