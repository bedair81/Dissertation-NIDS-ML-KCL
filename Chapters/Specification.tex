\chapter{Specification}

\section{Data Specification}
\begin{itemize}
    \item \textbf{Data Format:} The CTU-13 and CICIDS2017 datasets will be used in CSV format.
    \item \textbf{Data Features:} Specific features to be used from these datasets will include network flow attributes like source and destination IPs, port numbers, protocol types, and packet and byte counts.
    \item \textbf{Data Labeling:} Custom scripts will be developed using the python pandas library to relabel the CTU-13 dataset to align with the CICIDS2017 labeling system.
\end{itemize}

\section{Model Specification}
\begin{itemize}
    \item \textbf{Algorithm Selection:} The project will exclusively use the RandomForestClassifier, chosen for its proven performance in classifying network intrusion datasets, as evidenced by prior research.
    \item \textbf{Feature Selection and Engineering:} A key aspect of the project will involve experimenting with different combinations of features from the CTU-13 dataset. This includes determining which features to include or omit to enhance the model's performance when applied to the CICIDS2017 dataset.
    \item \textbf{Data Relabeling:} The CTU-13 dataset will be relabeled to align with the labeling schema of CICIDS2017, ensuring consistency in attack categorization and facilitating effective training and testing.
    \item \textbf{Hyperparameter Tuning:} Fine-tuning the RandomForestClassifier's hyperparameters, such as the number of trees, maximum depth of trees, and criteria for splitting, to optimize the model's performance for the specific task of detecting botnet attacks.
    \item \textbf{Model Validation:} Implementing validation strategies, such as K-fold cross-validation, to assess the model's performance and ensure its reliability and robustness in detecting network intrusions.
\end{itemize}

\section{Testing and Evaluation Specification}
\begin{itemize}
    \item \textbf{Testing Dataset:} The model will be tested on a separate subset of the CICIDS2017 dataset not used during the training phase.
    \item \textbf{Performance Metrics:} Metrics such as accuracy, precision, recall, F1-score, and ROC-AUC will be used for evaluation.
    \item \textbf{Generalization Assessment:} The model's performance on both datasets will be compared to assess its generalization capability.
\end{itemize}

\section{Explainability Framework}
\begin{itemize}
    \item \textbf{SHAP Integration:} The SHAP library will be integrated for model explainability, focusing on feature contribution to the model's predictions.
    \item \textbf{Interpretation Methodology:} Techniques like SHAP value plots and feature importance charts will be used to interpret the model.
\end{itemize}

\section{Technical Environment}
\begin{itemize}
    \item \textbf{Development Tools:} Python 3.x will be used along with libraries such as Pandas, NumPy, Scikit-learn, and SHAP.
    \item \textbf{Computational Resources:} The model will be trained using my PC at home running an AMD 5700G and a Nvidia RTX 3070 ti GPU.
\end{itemize}

\section{Compliance and Standards}
\begin{itemize}
    \item \textbf{Code Standards:} Adherence to PEP 8 standards for Python code.
    \item \textbf{Documentation:} Comprehensive documentation of the code, model, and results will be maintained.
\end{itemize}

\section{Conclusion}
The specifications outlined here provide a detailed roadmap for the implementation of the project. By adhering to these specifications, the project aims to develop a robust ML-based NIDS that is not only effective in detecting network intrusions but also transparent and understandable in its decision-making process.
