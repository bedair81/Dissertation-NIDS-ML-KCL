\chapter{Background}

\section{Machine Learning: An Overview}
Machine learning, a subset of artificial intelligence, involves the development of algorithms that enable computers to learn and make decisions from data. At its core, ML is about creating models that can process large datasets, identify patterns, and make predictions or decisions based on those patterns. ML can be broadly categorized into supervised learning, unsupervised learning, and reinforcement learning.

Supervised learning involves training a model on a labeled dataset, where the desired output is known. The model learns to map inputs to outputs, making it suitable for classification and regression tasks. Unsupervised learning, on the other hand, deals with unlabeled data. The goal here is to discover underlying patterns or structures in the data, which is useful for clustering and association tasks. Reinforcement learning involves training models to make sequences of decisions by rewarding or penalizing them for the actions they take in an environment.

ML algorithms range from simple linear regression to complex deep learning networks. The choice of algorithm depends on the nature of the task, the size and type of data, and the desired outcome. In the context of NIDS, ML algorithms must be capable of processing high-volume, high-dimensional data and providing accurate, real-time analysis.

\section{Network Intrusion Detection Systems}
Network Intrusion Detection Systems (NIDS) are essential for ensuring network security, tasked with monitoring network 
traffic for signs of malicious activities and potential threats. Traditional NIDS primarily rely on signature-based 
methods, which compare network traffic against a database of known attack patterns. However, the rapid evolution of 
cyber threats has necessitated the development of more advanced, adaptable systems. This shift has led to the emergence 
of machine learning-based NIDS, which represent the state-of-the-art in intrusion detection research. Unlike their 
traditional counterparts, ML-based NIDS learn to identify and predict threats by analyzing patterns in network traffic 
data, making them more effective against novel or sophisticated attacks.

Datasets such as CICIDS2017 and CTU-13 play a pivotal role in this context. The CICIDS2017 dataset, for instance, 
is a comprehensive collection of network traffic features that includes a wide range of attack scenarios, making it 
an invaluable resource for training and evaluating ML-based NIDS models~\cite{stiawan2020cicids}. Similarly, the CTU-13 
dataset, with its focus on botnet behavior, provides unique insights into more covert forms of network threats, 
further enhancing the training and testing of these advanced systems~\cite{garcia2014empirical}. These datasets are crucial for 
the ongoing development of ML-based NIDS, enabling researchers to refine algorithms and models to keep pace with the 
constantly evolving landscape of network security threats.

\section{The Basis of Machine Learning in NIDS}
Machine Learning (ML) has revolutionized the field of network intrusion detection by introducing models that can learn and adapt from data. Unlike traditional rule-based systems, ML-based NIDS leverage algorithms to analyze and learn from network traffic patterns, enabling them to detect novel or sophisticated cyber threats. Supervised learning models, such as Random Forests and Neural Networks, are particularly prevalent in this domain. They are trained on labeled datasets like CICIDS2017, where each network event is tagged as either normal or an attack, allowing the model to learn distinguishing features of various attack types~\cite{sharafaldin2018toward}.

The effectiveness of ML in NIDS is largely attributed to its ability to process and analyze large volumes of data, identifying patterns and anomalies that might be indicative of a network intrusion. By utilizing a range of features from network traffic, such as IP addresses, port numbers, protocol types, and payload characteristics, ML models can discern between benign and malicious activities with high accuracy. This capability is particularly crucial in the context of zero-day attacks and advanced persistent threats (APTs), where traditional signature-based methods fall short.

\section{Challenges in ML-based NIDS}
Despite their advantages, ML-based NIDS face several challenges. One of the primary concerns is the explainability of their decisions. As these systems often operate as black boxes, understanding the rationale behind specific alerts can be challenging, which is crucial for trust and actionable response in security operations~\cite{garcia2014empirical}. Another significant challenge is the fast concept drift in network environments. Cyber threats are constantly evolving, and models trained on historical data may quickly become outdated, necessitating continuous retraining and adaptation.

The issue of false positives and false negatives also presents a significant challenge in ML-based NIDS.\@High rates of false positives can lead to alert fatigue, where security analysts become desensitized to warnings, potentially overlooking genuine threats. Conversely, false negatives, where actual attacks are not detected, can have dire consequences for network security. Balancing sensitivity and specificity in ML models is therefore a critical aspect of NIDS design.

\section{Current State of ML-based NIDS}
The current state of ML-based NIDS is a dynamic landscape of ongoing research and development. Researchers are actively exploring ways to improve the accuracy, efficiency, and explainability of these systems. The use of advanced deep learning techniques and the integration of explainability frameworks like SHAP (SHapley Additive exPlanations) are current trends in this field. These methods aim to provide more transparent and interpretable models, thereby bridging the gap between high detection performance and the ability to understand and trust the model's decisions.

Recent advancements in ML, such as the development of convolutional neural networks (CNNs) and recurrent neural networks (RNNs), have shown promise in enhancing the detection capabilities of NIDS.\@These models are adept at processing sequential data and identifying complex patterns, which are common in network traffic. The integration of these advanced models into NIDS is an area of active research, with the potential to significantly improve the detection of sophisticated cyber threats.

\section{Future Directions}
Looking forward, the field of ML-based NIDS is moving towards more adaptive and autonomous systems. The integration of active learning and online learning approaches, where the system can update its model in real-time based on new data, is a promising direction. This approach could address the challenge of concept drift and ensure that NIDS remain effective against the ever-evolving landscape of network threats. Additionally, the exploration of unsupervised and semi-supervised learning methods could offer new ways to detect previously unknown types of attacks, further enhancing the robustness of NIDS.\@

The future of ML-based NIDS also lies in the integration of these systems with other components of network security, such as firewalls, intrusion prevention systems (IPS), and security information and event management (SIEM) systems. Such integration would enable a more holistic approach to network security, where different components work in synergy to provide enhanced protection against a wide range of cyber threats.