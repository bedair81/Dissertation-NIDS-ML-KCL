\chapter{Requirements}

\section{Data Preparation and Preprocessing}
The project requires meticulous data preparation and preprocessing to ensure the datasets are suitable for ML modeling. This involves:

\begin{itemize}
    \item Downloading and assembling the CTU-13 and CICIDS2017 datasets.
    \item Cleaning and preprocessing the data, which includes handling missing values, removing irrelevant features, and normalizing the data.
    \item Relabeling the CTU-13 dataset to match the labeling schema of CICIDS2017, ensuring consistency in attack categorization.
    \item Splitting the datasets into training and testing sets, with a focus on maintaining a balanced representation of different attack types.
\end{itemize}

\section{Model Development and Training}
The core of this project involves developing and training a machine learning model. The requirements for this phase include:

\begin{itemize}
    \item Selecting appropriate ML algorithms, with an initial focus on RandomForestClassifier due to its effectiveness in classification tasks.
    \item Training the model on the CTU-13 dataset, tuning hyperparameters to optimize performance.
    \item Implementing cross-validation techniques to ensure the model's generalizability and robustness.
\end{itemize}

\section{Model Testing and Evaluation}
Post-training, the model will be rigorously tested and evaluated:

\begin{itemize}
    \item Testing the trained model on the CICIDS2017 dataset to assess its effectiveness in detecting botnet attacks.
    \item Evaluating the model's performance using metrics such as accuracy, F1-score, precision, recall, and the confusion matrix.
    \item Analyzing the model's ability to generalize from CTU-13 to CICIDS2017, identifying any overfitting or underfitting issues.
\end{itemize}

\section{Explainability and Interpretation}
A key aspect of this project is to ensure the explainability of the ML model:

\begin{itemize}
    \item Utilizing the SHAP library to interpret the model's decisions, providing insights into feature importance and decision logic.
    \item Documenting and explaining the model's predictions, particularly in distinguishing between different types of network attacks.
\end{itemize}

\section{Technical Requirements}
The project will leverage various tools and technologies:

\begin{itemize}
    \item Python programming language, with libraries such as Pandas, NumPy, Scikit-learn for ML modeling, and SHAP for explainability.
    \item An appropriate development environment, possibly Jupyter Notebooks or a Python IDE, for coding and testing.
    \item Access to computational resources capable of handling large datasets and intensive computations.
\end{itemize}

\section{Additional Considerations}
If time permits, the project may explore:

\begin{itemize}
    \item Extending the analysis to additional datasets like UGR16, CICIDS2018, or UNSW-NB15 for further validation of the model's capabilities.
    \item Investigating different ML algorithms or deep learning approaches for comparative analysis.
\end{itemize}

\section{Conclusion}
This project aims to contribute to the field of network intrusion detection by leveraging machine learning for effective attack detection and providing clear explanations for the model's decisions. The successful completion of these requirements will demonstrate the potential of ML in enhancing network security measures.
