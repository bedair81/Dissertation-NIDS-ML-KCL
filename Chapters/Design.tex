\chapter{Design}

\section{Overview}
The design of the ML-based Network Intrusion Detection System (NIDS) is structured to effectively utilize machine learning for the detection of network intrusions, specifically focusing on botnet attacks within the CICIDS2017 dataset, using a model trained on the CTU-13 dataset. The process encompasses several stages, from data preprocessing to model evaluation and explainability analysis using SHAP.\@

\section{Data Preprocessing Design}
Data preprocessing is a critical step in ensuring the quality and consistency of the datasets used for training and testing the model. The design includes:

\begin{itemize}
    \item \textbf{Data Cleaning:} Initial cleaning of both CTU-13 and CICIDS2017 datasets to handle missing values and remove irrelevant features.
    \item \textbf{Feature Selection:} Careful selection of relevant network flow attributes, such as IP addresses, port numbers, and packet details.
    \item \textbf{Data Relabeling:} Utilizing custom Python scripts to modify the labeling of the CTU-13 dataset to align with the CICIDS2017 dataset's format.
    \item \textbf{Normalization:} Standardizing the scale of the data features to improve the performance of the machine learning model.
\end{itemize}

\section{Machine Learning Model Design}
The design of the machine learning model involves:

\begin{itemize}
    \item \textbf{Model Selection:} Employing the RandomForestClassifier due to its proven effectiveness in similar classification tasks.
    \item \textbf{Training Process:} Training the model on the preprocessed CTU-13 dataset with a focus on detecting botnet attacks.
    \item \textbf{Hyperparameter Optimization:} Tuning model parameters to find the optimal configuration for the best performance.
    \item \textbf{Validation Strategy:} Implementing cross-validation techniques to assess the model's effectiveness and prevent overfitting.
\end{itemize}

\section{Testing and Evaluation Design}
The testing and evaluation phase is designed to assess the model's performance and generalization capabilities:

\begin{itemize}
    \item \textbf{Testing Dataset:} Applying the trained model to the CICIDS2017 dataset to evaluate its ability to detect botnet attacks.
    \item \textbf{Performance Metrics:} Utilizing accuracy, precision, recall, F1-score, and other relevant metrics for a comprehensive evaluation.
    \item \textbf{Comparative Analysis:} Analyzing the model's performance on both datasets to identify any discrepancies and areas for improvement.
\end{itemize}

\section{Explainability and Interpretation Design}
To ensure the transparency and understandability of the model's decisions:

\begin{itemize}
    \item \textbf{SHAP Integration:} Incorporating the SHAP library to provide insights into how different features influence the model's predictions.
    \item \textbf{Decision Explanation:} Detailed analysis of SHAP values to interpret the model's behavior, especially in distinguishing between normal traffic and botnet attacks.
\end{itemize}

\section{Technical and Environmental Setup}
The project will be implemented in a Python environment with the following considerations:

\textbf{Development Environment:} Utilizing Python 3.x and libraries such as Pandas, NumPy, Scikit-learn, and SHAP.\@
\textbf{Hardware Resources:} Leveraging a personal computer equipped with an AMD 5700G CPU and an Nvidia RTX 3070 ti GPU for model training and testing.

\section{Conclusion}
The design of this ML-based NIDS project is structured to effectively train a model on the CTU-13 dataset and test its performance on the CICIDS2017 dataset, with a focus on detecting botnet attacks. The integration of SHAP for explainability ensures that the model's decisions are transparent and interpretable, contributing to the field of network security.
