Here is the revised introduction with the requested changes:

\chapter{Introduction}

In recent years, the increasing prevalence and sophistication of cyber attacks have necessitated the development of robust network intrusion detection systems (NIDS). Traditional signature-based NIDS have struggled to keep pace with the evolving threat landscape, leading researchers to explore machine learning approaches for detecting novel and previously unseen attacks~\cite{marchetti2016analysis}. However, the effectiveness of machine learning-based NIDS is heavily dependent on the quality and representativeness of the datasets used for training and evaluation~\cite{engelen2021troubleshooting}.

This paper focuses on comparing two widely used datasets for network intrusion detection: CTU13~\cite{garcia2014empirical} and CICIDS2017~\cite{sharafaldin2018toward}. The CTU13 dataset contains real botnet traffic mixed with normal traffic and is often used for evaluating the performance of NIDS in detecting botnets. On the other hand, the CICIDS2017 dataset is a more recent and comprehensive dataset that includes a wide range of modern attacks such as DoS, DDoS, brute force, XSS, SQL injection, and infiltration~\cite{sharafaldin2018toward}. The main objective of this research is to investigate the effectiveness of using machine learning models trained on the CICIDS2017 dataset to detect botnet attacks in the CTU13 dataset. Specifically, this paper trains two popular machine learning classifiers, namely Random Forest and Support Vector Machine (SVM), on the CICIDS2017 dataset and evaluates their performance on detecting botnet attacks in CTU13. This approach allows the assessment of the generalizability and transferability of the learned features and patterns from one dataset to another.

However, it is important to note that machine learning-based NIDS are not without their challenges. Sommer and Paxon~\cite{sommer2010outside} highlighted the limitations of using machine learning for network intrusion detection, particularly in terms of the difficulty in obtaining representative training data and the potential for adversarial attacks. Pierazzi et al.~\cite{pierazzi2020intriguing} further explored the intriguing properties of adversarial attacks in the problem space of machine learning-based security systems.

This paper draws upon the recommendations of Arp et al.\cite{arp2022and} on the dos and don'ts of machine learning for security to guide the experimental design and analysis. In addition to the Random Forest and SVM classifiers, this paper also explores the use of explainable AI techniques, specifically SHAP (SHapley Additive exPlanations)\cite{lundberg2017unified}, to gain insights into the decision-making process of the trained models. SHAP provides a unified framework for interpreting predictions and helps identify the most important features contributing to the models' decisions. By applying SHAP to the trained classifiers, this paper aims to understand the key patterns and characteristics that distinguish botnet traffic from normal traffic in both the CTU13 and CICIDS2017 datasets.

Furthermore, this paper delves into the comparative analysis of the network flow features extracted from the CTU13 and CICIDS2017 datasets. It examines the similarities and differences in the makeup of these datasets, considering that CTU13 contains real network traffic while CICIDS2017 is synthetically generated. By exploring the statistical properties and distributions of various flow features, such as duration, packet size, inter-arrival times, and protocol usage, this paper aims to uncover the inherent characteristics that differentiate real and simulated network traffic. This analysis provides valuable insights into the challenges and limitations of using synthetic datasets for training NIDS models and highlights the importance of considering dataset biases when evaluating their performance.

The inclusion of the SVM classifier in this study allows for a more comprehensive evaluation of the transferability and generalizability of machine learning models across different datasets. By comparing the performance of Random Forest and SVM, this paper can assess the robustness and adaptability of these classifiers in detecting botnet attacks in a real-world scenario using the CTU13 dataset. Additionally, the application of SHAP to both classifiers enables the identification and comparison of the most influential features for each model, providing a deeper understanding of the underlying patterns and characteristics that contribute to accurate botnet detection.

The remainder of this paper is structured as follows: Section 2 provides a comprehensive background and literature review of related work in the field of machine learning-based NIDS.\@ It discusses the state-of-the-art approaches, their strengths, and limitations, setting the context for this research. Section 3 presents the specification and design of the experimental setup, including the detailed description of the CTU13 and CICIDS2017 datasets, their collection methodologies, attack scenarios, and feature sets. It also outlines the data preprocessing and feature engineering steps undertaken to prepare the data for machine learning. Section 4 focuses on the implementation aspects of this study, including the training and optimization of the Random Forest and SVM classifiers, as well as the application of SHAP for model interpretability. Section 5 presents the evaluation of the proposed approach, discussing the performance metrics, comparative analysis of the classifiers, and the insights gained from SHAP explanations. It also highlights the key findings from the flow-level analysis of the datasets and their implications for NIDS development and evaluation. Section 6 explores the legal, social, ethical, and professional issues related to the use of machine learning in network intrusion detection, addressing concerns such as data privacy, fairness, and the potential impact on society. Finally, Section 7 concludes the paper by summarizing the main contributions, discussing the limitations of this study, and outlining potential future research directions to further advance the field of machine learning-based NIDS.\@