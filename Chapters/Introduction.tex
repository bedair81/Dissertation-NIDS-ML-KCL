\chapter{Introduction}\label{chap:introduction}

As the digital landscape becomes increasingly interconnected, safeguarding computer network security is crucial. The continuous evolution of cyber threats challenges the efficacy of conventional signature-based Network Intrusion Detection Systems (NIDS), which struggle to identify and mitigate novel and complex attacks~\cite{marchetti2016analysis}. In response, the research community has pivoted towards machine learning strategies that recognise unseen intrusions by analysing network traffic data for patterns and anomalies~\cite{buczak2015survey}.

A pivotal factor in the success of machine learning-driven NIDS is the authenticity and comprehensiveness of the datasets utilised for training and evaluation purposes~\cite{engelen2021troubleshooting}. Prominent among these are the CTU13 dataset, encompassing genuine botnet traffic within normal traffic~\cite{garcia2014empirical}, and the CICIDS2017 dataset, which encompasses a broad spectrum of contemporary attack vectors~\cite{sharafaldin2018toward}. Despite their widespread usage, the extent to which machine learning models can generalise learned patterns from one dataset to accurately predict intrusions in another dataset still needs to be explored.

This dissertation endeavours to bridge this knowledge gap by meticulously examining the effectiveness of machine learning algorithms, specifically trained on the CICIDS2017 dataset, in detecting botnet activities within the CTU13 dataset. Employing Random Forest and Support Vector Machine (SVM) classifiers trained on CICIDS2017, we evaluate their performance on the CTU13 dataset. This methodology facilitates an investigation into the transferability and applicability of learned features and patterns across distinct datasets, mirroring a scenario wherein a model trained on a particular dataset is applied to safeguard against intrusions in diverse network environments.

Guided by Arp et al.'s recommendations for the application of machine learning in cybersecurity contexts~\cite{arp2022and}, this research incorporates explainable AI techniques, notably SHAP (SHapley Additive exPlanations)~\cite{lundberg2017unified}, to elucidate our trained models' decision-making processes. Identifying critical features for the detection of botnet traffic enables us to elucidate distinctive patterns and characteristics differentiating malicious from benign network flows.

Furthermore, this dissertation conducts a detailed comparative analysis of network flow features derived from the CTU13 and CICIDS2017 datasets. By examining the statistical properties and distribution of various features, such as flow duration, packet counts, and inter-arrival times, we aim to discern the discrepancies between real and synthetically generated network traffic. This comparison illuminates synthetic datasets' inherent challenges and constraints in training NIDS models, underscoring the imperative of acknowledging dataset biases during performance assessments.

Incorporating both Random Forest and SVM classifiers, this study facilitates a thorough evaluation of model transferability and generalizability. Through performance comparison and prediction interpretation via SHAP, we derive significant insights into the resilience and versatility of these algorithms in identifying botnet attacks across different datasets.

This dissertation is structured as follows: Section 2 provides a foundational understanding of the CTU13 and CICIDS2017 datasets, machine learning classifiers, and explainable AI techniques. Section 3 reviews pertinent literature in the domain of machine learning-based NIDS.\@ Section 4 outlines the specifications and design of our experimental framework, whereas Section 5 elaborates on the implementation. Section 6 discusses the evaluation outcomes and their implications. Section 7 deliberates on the legal, social, ethical, and professional considerations of deploying machine learning in network intrusion detection. The dissertation concludes with Section 8, summarising our findings and suggesting avenues for future research.