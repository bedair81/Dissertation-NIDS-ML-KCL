\chapter{Introduction}\label{chap:introduction}

In an increasingly interconnected digital landscape, safeguarding computer network security is of paramount importance. The relentless evolution of cyber threats challenges the effectiveness of conventional signature-based Network Intrusion Detection Systems (NIDS), which often fail to detect novel and sophisticated attacks~\cite{marchetti2016analysis}. Consequently, the research community has shifted towards machine learning approaches that identify unseen intrusions by analysing network traffic for patterns and anomalies~\cite{buczak2015survey}.

The efficacy of machine learning-driven NIDS depends significantly on the authenticity and comprehensiveness of the datasets employed for training and evaluation. The CTU13 and CICIDS2017 datasets stand out as prominent examples, providing genuine botnet traffic within regular network flows and a diverse array of contemporary attack vectors, respectively~\cite{garcia2014empirical, sharafaldin2018toward}. Despite their extensive use, a critical research gap persists: understanding how machine learning models can generalise patterns learned from one dataset to accurately detect intrusions in another remains underexplored.

This dissertation investigates the effectiveness of machine learning algorithms—specifically Random Forest and Support Vector Machine (SVM) classifiers—trained on the CICIDS2017 dataset, in identifying botnet activities within the CTU13 dataset. By evaluating their performance on CTU13, we assess the transferability and applicability of learned features across distinct datasets. This approach mirrors real-world scenarios where models trained on specific datasets must protect diverse network environments against intrusions.

Following Arp et al.’s guidelines for applying machine learning in cybersecurity~\cite{arp2022and}, this study integrates explainable AI techniques, notably SHAP (SHapley Additive exPlanations)~\cite{lundberg2017unified}, to clarify the decision-making processes of our models. By pinpointing critical features for botnet detection, we reveal the distinctive patterns that differentiate malicious from benign network flows, enhancing interpretability and trust in the results.

Additionally, this dissertation conducts a comparative analysis of network flow features from the CTU13 and CICIDS2017 datasets. Examining statistical properties—such as flow duration, packet counts, and inter-arrival times—we highlight differences between real and synthetic traffic. This analysis underscores the limitations of synthetic datasets in training NIDS models and emphasises the importance of addressing dataset biases in performance evaluations~\cite{sommer2010outside}.

Through the use of Random Forest and SVM classifiers, this study provides a thorough evaluation of model transferability and generalisability. Performance comparisons, coupled with SHAP-based interpretation, offer valuable insights into the robustness and versatility of these algorithms in detecting botnet attacks across datasets, advancing their potential application in cybersecurity.

This dissertation is structured as follows: Section 2 introduces the CTU13 and CICIDS2017 datasets, machine learning classifiers, and explainable AI methods. Section 3 reviews existing literature on machine learning-based NIDS.\@ Section 4 details the experimental framework’s design, while Section 5 describes its implementation. Section 6 presents the evaluation results and their implications. Section 7 addresses the legal, social, ethical, and professional considerations of deploying machine learning in intrusion detection. Finally, Section 8 concludes with a summary of findings and recommendations for future research.