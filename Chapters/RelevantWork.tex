\chapter{Relevant Work}

This chapter comprehensively reviews related work in machine learning-based network intrusion detection. It covers state-of-the-art approaches for botnet detection using the CTU13 dataset, intrusion detection using the CICIDS2017 dataset, and the application of Random Forest and Support Vector Machines for intrusion detection tasks. Furthermore, it explores the use of explainable AI techniques, such as SHAP, to provide insights into the decision-making process of machine learning models in the context of intrusion detection. The chapter also discusses the challenges and limitations associated with using synthetic datasets and the importance of considering dataset biases and the dynamic nature of network traffic when developing and evaluating intrusion detection systems.

\section{Botnet Detection using CTU13 Dataset}

Several studies have utilized the CTU13 dataset for developing and evaluating botnet detection systems. Chowdhury et al.~\cite{chowdhury2017botnet} proposed a graph-based approach for botnet detection using the CTU13 dataset. They constructed a graph representation of the communication patterns among botnet-infected hosts and applied graph analysis techniques to identify botnets. The authors extracted features such as in-degree, out-degree, and betweenness centrality from the graph and used them to train a Random Forest classifier. Their approach achieved an accuracy of 99.86\% in detecting botnets, demonstrating the potential of leveraging graph-based features for botnet detection. In contrast, this dissertation investigates the transferability of models trained on the CICIDS2017 dataset to detect botnet attacks in the CTU13 dataset, exploring the generalization capabilities of machine learning classifiers across different datasets.

Pektaş and Acarman~\cite{pektacs2019deep} applied deep learning techniques to the CTU13 dataset for botnet detection. They used a convolutional neural network (CNN) to learn discriminative features from raw network traffic data. The authors converted the traffic data of the CTU13 dataset into grayscale images through preprocessing and then fed them as input to the CNN.\@ The proposed model achieved an accuracy of 99.99\% and a false positive rate of 0.01\% in detecting botnet traffic, showcasing the effectiveness of deep learning in capturing complex patterns and behaviours associated with botnets. While deep learning approaches have shown promising results, this dissertation explores the potential of traditional machine learning classifiers, such as Random Forest and Support Vector Machines, in detecting botnet attacks and investigates their transferability across datasets.

\section{Intrusion Detection using CICIDS2017 Dataset}

Many researchers have widely used the CICIDS2017 dataset to evaluate intrusion detection systems. Ustebay et al.~\cite{ustebay2018intrusion} proposed an intrusion detection system based on a multi-layer perceptron (MLP) classifier using the CICIDS2017 dataset. They performed extensive preprocessing on the dataset, including handling missing values, encoding categorical features, and scaling numerical features. The authors also applied recursive feature elimination with random forest to select the most relevant features for intrusion detection. Their MLP-based approach achieved an accuracy of 97.29\% and an F1-score of 97.30\%, demonstrating the potential of neural network-based models for network intrusion detection. In this dissertation, we leverage the CICIDS2017 dataset to train machine learning classifiers and investigate their transferability to detect botnet attacks in the CTU13 dataset, focusing on the generalization capabilities of the learned patterns and features.

Aksu and Aydin~\cite{aksu2018detecting} conducted a comparative study of different machine learning algorithms for network intrusion detection using the CICIDS2017 dataset. They evaluated the performance of decision trees, random forests, and support vector machines. The authors preprocessed the dataset by removing redundant records, handling missing values, and applying normalization techniques. They also performed feature selection using information gain and correlation-based methods. The experimental results showed that random forests outperformed other algorithms, achieving an accuracy of 99.77\% and a false positive rate of 0.04\%, highlighting the effectiveness of ensemble learning methods for intrusion detection. This dissertation takes inspiration from their findings and employs Random Forest as one of the classifiers to investigate its transferability across datasets.

\section{Random Forest for Intrusion Detection}

Farnaaz and Jabbar~\cite{farnaaz2016random} utilized Random Forest for intrusion detection tasks and presented a model based on it. They evaluated the model's performance on the NSL-KDD dataset. They performed feature selection using the Chi-square test to identify the most relevant features for intrusion detection, then proceeded to train the Random Forest classifier using the selected features. The proposed model achieved an accuracy of 99.67\% and a false positive rate of 0.06\%, demonstrating Random Forest's effectiveness in detecting various types of network attacks. While their work focused on a single dataset, this dissertation explores the transferability of Random Forest classifiers trained on the CICIDS2017 dataset to detect botnet attacks in the CTU13 dataset, investigating the generalization capabilities of the learned patterns across different network environments.

Belouch et al.~\cite{belouch2018performance} applied Random Forest to the CICIDS2017 dataset for network intrusion detection. They compared the performance of Random Forest with other machine learning algorithms, including Decision Tree, Naive Bayes, and k-Nearest Neighbors. The authors preprocessed the dataset by removing redundant records and handling missing values. They also analyzed the importance of different features in the dataset using the Random Forest feature importance measure. The experimental results showed that Random Forest outperformed other algorithms, achieving an accuracy of 99.98\% and a false positive rate of 0.01\%, further validating the effectiveness of Random Forest for intrusion detection. This dissertation builds upon their findings by employing Random Forest as one of the classifiers. It investigates its transferability to detect botnet attacks in a different dataset, exploring the robustness and adaptability of the learned patterns.

\section{Support Vector Machines for Intrusion Detection}

Intrusion detection tasks have extensively utilized Support Vector Machines (SVM). Kabir et al.~\cite{kabir2017network} proposed an SVM-based intrusion detection system and evaluated its performance on the NSL-KDD dataset. They used a genetic algorithm for feature selection to identify the most discriminative features for intrusion detection. The authors also optimized the SVM parameters using grid search to improve the model's performance. The proposed system achieved an accuracy of 99.91\% and a detection rate of 99.93\%, showcasing SVM's effectiveness in detecting various types of network attacks. In contrast to their work, this dissertation focuses on the transferability of SVM classifiers trained on the CICIDS2017 dataset to detect botnet attacks in the CTU13 dataset, exploring the generalization capabilities of the learned patterns across different datasets.

Teng et al.~\cite{teng2017svm} applied SVM with different kernel functions for intrusion detection on the CICIDS2017 dataset. They compared the performance of linear, polynomial, and radial basis function (RBF) kernels. The authors preprocessed the dataset by removing redundant records and scaling the features. They also applied principal component analysis (PCA) for dimensionality reduction. The experimental results showed that the RBF kernel achieved the highest accuracy of 97.80\% and a low false positive rate of 0.12\%, highlighting the importance of selecting appropriate kernel functions and tuning the SVM parameters for optimal performance. This dissertation takes inspiration from their findings and employs SVM as one of the classifiers to investigate its transferability in detecting botnet attacks across different datasets, exploring the robustness and adaptability of the learned patterns.

\section{Explainable AI Techniques for Intrusion Detection}

Explainable AI techniques have gained attention in intrusion detection to provide insights into the decision-making process of machine learning models. Warnecke et al.~\cite{warnecke2020evaluating} evaluated various explanation methods for deep learning-based intrusion detection systems, including SHAP\@ They applied SHAP to a convolutional neural network (CNN) trained on the NSL-KDD dataset and analyzed the importance of different features in the model's predictions. The authors demonstrated that SHAP can provide meaningful insights into the decision-making process of deep learning models and help identify the most influential features for detecting specific attack types. This dissertation takes inspiration from their approach and employs SHAP to interpret the predictions of Random Forest and SVM classifiers trained on the CICIDS2017 dataset, providing insights into the transferability of the learned patterns and the most relevant features for detecting botnet attacks in the CTU13 dataset.

Amarasinghe et al.~\cite{amarasinghe2018toward} employed SHAP to interpret the predictions of a deep learning-based network intrusion detection system. They trained a deep neural network on the NSL-KDD dataset and applied SHAP to explain the model's predictions. The authors analyzed the SHAP values to understand the impact of different features on the model's decisions and identified the most discriminative features for detecting specific types of attacks. Their study highlights the potential of SHAP in providing interpretable explanations for deep learning-based intrusion detection systems. In this dissertation, we leverage SHAP to interpret the predictions of traditional machine learning classifiers, such as Random Forest and SVM, and investigate the transferability of the learned patterns and the most relevant features for detecting botnet attacks across different datasets.

Mane and Rao~\cite{mane2021explaining} used SHAP to explain the predictions of a random forest classifier for network intrusion detection. They trained the classifier on the NSL-KDD dataset and applied SHAP to interpret the model's predictions. The authors visualized the SHAP values to understand the contribution of each feature towards the model's output and identified the most critical features for detecting various types of network attacks. Their study demonstrates the effectiveness of SHAP in providing interpretable explanations for ensemble learning methods like random forests. This dissertation builds upon their findings by employing SHAP to interpret the predictions of Random Forest and SVM classifiers trained on the CICIDS2017 dataset. It investigates the transferability of the learned patterns and the most relevant features for detecting botnet attacks in the CTU13 dataset.

\section{Challenges and Limitations}

Intrusion detection system developers and evaluators frequently employ the CTU13 and CICIDS2017 datasets. Nonetheless, it is necessary to recognize their limitations. Sommer and Paxson~\cite{sommer2010outside} have addressed the challenges of utilizing synthetic datasets for network intrusion detection. They contend that these datasets may only partially capture the intricacy and variety of network traffic and require additional contextual information. Furthermore, the authors stress the necessity of more authentic and comprehensive datasets that consider the operational aspects and deployment hurdles of intrusion detection systems. This dissertation acknowledges these limitations and emphasizes the importance of considering dataset biases and the representativeness of training data when developing and evaluating intrusion detection systems.

Furthermore, Sommer and Paxson~\cite{sommer2010outside} highlight the importance of using representative and unbiased datasets for training machine learning models in network intrusion detection. They discuss the challenges posed by the dynamic nature of network traffic and the constant evolution of attack patterns, which can impact the long-term performance of the trained models. Buczak and Guven~\cite{buczak2015survey} provide a comprehensive survey of machine learning and data mining methods for cyber security intrusion detection. They discuss the considerations and challenges in applying machine learning techniques to network intrusion detection, including selecting appropriate algorithms, feature engineering, and model evaluation. This dissertation considers these challenges and aims to address them by investigating the transferability of machine learning models across different datasets and employing explainable AI techniques to gain insights into the decision-making process.

\section{Summary and Positioning}

This chapter reviews relevant work in machine learning-based network intrusion detection, focusing on botnet detection using the CTU13 dataset, intrusion detection using the CICIDS2017 dataset, and the application of Random Forests and Support Vector Machines for intrusion detection tasks. While previous studies have demonstrated the effectiveness of various machine learning algorithms in detecting network attacks, this dissertation explores a novel aspect by investigating machine learning models' transferability and generalization capabilities across different datasets.

This dissertation takes inspiration from existing approaches that have utilized Random Forest and Support Vector Machines for intrusion detection and applies these classifiers to the CICIDS2017 dataset. However, this work goes beyond previous studies by evaluating the performance of these classifiers in detecting botnet attacks in the CTU13 dataset, which represents a different network environment. This dissertation aims to assess the robustness and adaptability of the learned patterns and features across datasets, simulating a real-world scenario where a model is trained on one dataset and deployed to detect intrusions in the wild.

Furthermore, while the use of explainable AI techniques, such as SHAP, has been explored in previous studies for interpreting the predictions of intrusion detection models, this dissertation employs SHAP in a novel context. Specifically, it leverages SHAP to provide insights into the transferability of learned patterns and the most relevant features for detecting botnet attacks when applying classifiers trained on the CICIDS2017 dataset to the CTU13 dataset. This aspect of the research distinguishes it from previous studies, as it focuses on understanding the transferability of the learned patterns and the key characteristics that distinguish malicious and benign traffic across different datasets.

In summary, this dissertation builds upon existing research by investigating machine learning models' transferability and generalization capabilities for network intrusion detection, emphasizing the importance of dataset representativeness and biases. By employing Random Forest and Support Vector Machines classifiers and leveraging explainable AI techniques in the context of cross-dataset transferability, this work aims to provide valuable insights into the robustness and adaptability of these algorithms in detecting botnet attacks across different datasets, contributing to the development of more effective and practical intrusion detection solutions.