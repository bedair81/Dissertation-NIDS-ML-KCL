\chapter{Relevant Work}

Several studies have utilised these datasets to develop and evaluate network intrusion detection systems. Chowdhury et al.~\cite{chowdhury2017botnet} proposed a graph-based approach for botnet detection using the CTU13 dataset. They constructed a graph representation of the communication patterns among botnet-infected hosts and applied graph analysis techniques to identify botnets. Their approach achieved high accuracy in detecting botnets and demonstrated the potential of leveraging graph-based features for botnet detection.

Pektaş and Acarman~\cite{pektacs2019deep} applied deep learning techniques to the CTU13 dataset for botnet detection. They used a convolutional neural network (CNN) to learn discriminative features from raw network traffic data. Their proposed model achieved high detection accuracy and showcased the effectiveness of deep learning in capturing complex patterns and behaviours associated with botnets.

Ustebay et al.~\cite{ustebay2018intrusion} proposed an intrusion detection system based on a multi-layer perceptron (MLP) classifier using the CICIDS2017 dataset. They performed extensive preprocessing and feature selection to optimise the input data for the MLP classifier. Their approach achieved high detection accuracy for various attack types in the dataset, demonstrating the potential of neural network-based models for network intrusion detection.

Aksu and Aydin~\cite{aksu2018detecting} conducted a comparative study of different machine learning algorithms for network intrusion detection using the CICIDS2017 dataset. They evaluated the performance of decision trees, random forests, and support vector machines. Their results showed that random forests outperformed other algorithms regarding accuracy and false-positive rates, highlighting the effectiveness of ensemble learning methods for intrusion detection.

Farnaaz and Jabbar~\cite{farnaaz2016random} proposed a Random Forest-based model for intrusion detection and evaluated its performance on the NSL-KDD dataset. They performed feature selection using the Chi-square test and trained the Random Forest classifier on the selected features. Their model achieved an accuracy of 99.67\% and a low false positive rate of 0.06\%, demonstrating Random Forest's effectiveness in detecting various types of network attacks.

Belouch et al.~\cite{belouch2018performance} applied Random Forest to the CICIDS2017 dataset for network intrusion detection. They compared the performance of Random Forest with other machine learning algorithms, including Decision Tree, Naive Bayes, and k-Nearest Neighbors. Their results showed that Random Forest outperformed other algorithms, achieving an accuracy of 99.98\% and a false positive rate of 0.01\%. They also analysed the importance of different features in the dataset and identified the most discriminative features for intrusion detection.

Kabir et al.~\cite{kabir2017network} proposed an SVM-based intrusion detection system and evaluated its performance on the NSL-KDD dataset. They used a genetic algorithm for feature selection and optimised the SVM parameters using grid search. Their proposed system achieved an accuracy of 99.91\% and a detection rate of 99.93\%, showcasing SVM's effectiveness in detecting various types of network attacks.

Teng et al.~\cite{teng2017svm} applied SVM with different kernel functions for intrusion detection on the CICIDS2017 dataset. They compared the performance of linear, polynomial, and radial basis function (RBF) kernels. Their results showed that the RBF kernel achieved the highest accuracy of 97.80\% and a low false positive rate of 0.12\%. They also highlighted the importance of selecting appropriate kernel functions and tuning the SVM parameters for optimal performance.

Warnecke et al.~\cite{warnecke2020evaluating} evaluated various explanation methods for deep learning-based intrusion detection systems, including SHAP They applied SHAP to a convolutional neural network (CNN) trained on the NSL-KDD dataset and analysed the importance of different features in the model's predictions. They demonstrated that SHAP can provide meaningful insights into the decision-making process of deep learning models and help identify the most influential features for detecting specific attack types.

Amarasinghe et al.~\cite{amarasinghe2018toward} employed SHAP to interpret the predictions of a deep learning-based NIDS.\@ They trained a deep neural network on the NSL-KDD dataset and applied SHAP to explain the model's predictions. They analysed the SHAP values to understand the impact of different features on the model's decisions and identified the most discriminative features for detecting specific types of attacks. Their study highlights the potential of SHAP in providing interpretable explanations for deep learning-based NIDS.\@

Mane and Rao~\cite{mane2021explaining} used SHAP to explain the predictions of a random forest classifier for network intrusion detection. They trained the classifier on the NSL-KDD dataset and applied SHAP to interpret the model's predictions. They visualised the SHAP values to understand the contribution of each feature towards the model's output and identified the most critical features for detecting various types of network attacks. Their study demonstrates the effectiveness of SHAP in providing interpretable explanations for ensemble learning methods like random forests.

While these studies demonstrate the utility of the CTU13 and CICIDS2017 datasets, it is crucial to acknowledge their limitations. Sommer and Paxson~\cite{sommer2010outside} discuss the challenges of using synthetic datasets for network intrusion detection. They argue that synthetic datasets may only partially capture the complexity and diversity of real-world network traffic and may need more crucial contextual information. They emphasise the need for more realistic and representative datasets considering intrusion detection systems' operational aspects and deployment challenges.

Furthermore, Sommer and Paxson~\cite{sommer2010outside} emphasise the significance of using representative and unbiased datasets for training machine learning models in network intrusion detection. They highlight the challenges posed by the dynamic nature of network traffic and the constant evolution of attack patterns, which can impact the long-term performance of the trained models. Buczak and Guven~\cite{buczak2015survey} provides a comprehensive survey of machine learning and data mining methods for cyber security intrusion detection. They discuss the considerations and challenges in applying machine learning techniques to network intrusion detection, including selecting appropriate algorithms, feature engineering, and model evaluation.

The choice of a machine learning classifier depends on various factors, such as the dataset's characteristics, the nature of the attacks, and the computational resources available. Buczak and Guven~\cite{buczak2015survey} provide a comprehensive survey of machine learning methods for cyber security intrusion detection. They discuss the strengths and limitations of different classifiers and emphasise the importance of selecting appropriate features, handling imbalanced data, and evaluating the performance of classifiers using relevant metrics.

Comparing the performance of different classifiers on multiple datasets provides valuable insights into their generalisation capabilities and transferability. Training classifiers on one dataset and testing them on another helps assess how well the learned patterns and features can be applied to detect intrusions in different network environments. This approach helps understand the robustness and adaptability of the classifiers across various scenarios.