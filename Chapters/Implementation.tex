\chapter{Implementation}

The following chapter delves into the details of the experiment's execution. In Section~\ref{sec:experimental-setup}, we provide a comprehensive description of the experimental setup, including an overview of the datasets, their usage in each classifier, and the overall design of the experiment. Section~\ref{sec:package-implementation} explores the implementation of the various packages, accompanied by code excerpts that highlight the essential functions of each component.

\section{Experimental Setup}\label{sec:experimental-setup}

We conduct three main experiements in this dissertation, with each experiment being split into a futher 3 subcategories. This section explains the setup for each experiment, highlighting the differnences in each and the reasoning behind the choices made.

\subsection{Pre-Processing}\label{subsec:pre-processing}

Before any classifiers can be trained, it is essential that the data being fed into them is processed correctly, after all, a classifier is only as good as the quality of the data it was trained on. This makes the preprocessing stage one of the most important, since any issues with the data at this early stage can have a major effect on the final results of each experiment.





\section{Package Implementation}\label{sec:package-implementation}

\subsection{relabelCICIDS2017.py}\label{subsec:relabelCICIDS2017.py}

\subsection{relabelCTU13.py}\label{subsec:relabelCTU13.py}